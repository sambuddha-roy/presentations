\documentclass{beamer}
\usetheme{umbc4}    
\usepackage[english]{babel}
\usepackage[latin1]{inputenc}
\usepackage{times}
\usepackage[T1]{fontenc}
\usepackage{amssymb}
\usepackage{color}

\newcommand{\si}{\mathrm{sim}}
\newcommand{\dist}{\mathrm{dist}}
\newcommand{\lsh}{\mathrm{LSH}}



\newcommand{\mba}{\mathrm{MBA}}
\newcommand{\lst}{\mathrm{LST}}
\newcommand{\bfs}{\mathrm{BFS}}
\newcommand{\amgm}{{\mathrm{AM-GM}}}
\newcommand{\lp}{\mathrm{LP}}
\newcommand{\lr}{\mathrm{LR}}
\newcommand{\sig}{\mathrm{sign}}

\newcommand{\leqs}{\leqslant}
\newcommand{\geqs}{\geqslant}
\newcommand{\tcg}[1]{\textcolor{green}{#1}}
\newcommand{\tcb}[1]{\textcolor{blue}{#1}}
\newcommand{\tcr}[1]{\textcolor{red}{#1}}


\newcommand{\opt}{\mathrm{OPT}}
\newcommand{\eps}{\epsilon}
\newcommand{\tum}{\mathrm{TUM}}
\newcommand{\np}{\mathrm{NP}}
\newcommand{\dtime}{\mathrm{DTIME}}
\newcommand{\mis}{\mathrm{MIS}}
\newcommand{\maxcut}{{\tt MAXCUT}}

\newcommand{\calS}{{\cal S}}
\newcommand{\calA}{{\cal A}}
\newcommand{\calB}{{\cal B}}
\newcommand{\calC}{{\cal C}}
\newcommand{\calD}{{\cal D}}
\newcommand{\calE}{{\cal E}}
\newcommand{\calF}{{\cal F}}


\newcommand{\calT}{{\cal T}}
\newcommand{\calR}{{\cal R}}


\newcommand{\mstaff}{{\tt MaxStaff}}
\newcommand{\eat}[1] {}


\newcommand{\alrt}[1]{{\color{alert} #1}}
\newcommand{\alrtl}[1]{{\color{alert}\large #1}}
\newcommand{\alrtL}[1]{{\color{alert}\Large #1}}
\newcommand{\struc}[1]{{\color{structure} #1}}
\newcommand{\strucL}[1]{{\color{structure}\Large #1}}
\newcommand{\strucl}[1]{{\color{structure}\large #1}}


\newcommand{\paintblue}[1]{$\tikz[baseline]{
            \node[fill=blue!20,anchor=base] (t1)
            {$ #1$ };
        }$ }

\newcommand{\paintred}[1]{$\tikz[baseline]{
            \node[fill=red!20,anchor=base] (t1)
            {$ #1$ };
        } $}

\newcommand {\lovasz}{{Lov\'{a}sz}}

\newcommand{\textred}[1]{{\textcolor{red}{#1}}}
\newcommand{\textblue}[1]{{\textcolor{blue}{#1}}}


\usepackage{tikz,tkz-berge}
\setbeamertemplate{background canvas}[vertical shading][bottom=green!20,top=yellow!30]
\setbeamertemplate{itemize item}{\scriptsize\raise1.25pt\hbox{\donotcoloroutermaths$\blacktriangleright$}}


\title[IISc., July 7, 2016] % (optional, use only with long paper titles)
{Common Inequalities in Computer Science}

\author{Sambuddha Roy, LinkedIn}

\begin{document}

\begin{frame}
  \titlepage
\end{frame}

\begin{frame}
\frametitle{Contents}
\begin{itemize}
\item Why inequalities?
\item Used in: almost everywhere..
\pause
\begin{itemize}
\item Approximation Algorithms \pause 
\item Optimization \pause
\item Machine Learning \pause
\item Mathematical Analysis \pause
\item ...
\pause
\item (Can almost be called the backbone of mathematics..)
\end{itemize}
\end{itemize}
\end{frame}

\begin{frame}
\frametitle{See it in real life}
\begin{itemize}
\item So you have an algorithm - how do you prove it to be optimal (or close to optimal)?
\pause
\item You have to show that there is a {\em lower bound} for the resources that the algorithm 
takes.
\pause
\item Essentially - prove inequalities!
\end{itemize}
\end{frame}

\begin{frame}
\frametitle{A simple inequality}
\begin{itemize}
\item Setting: We are given $n$ positive real numbers $x_1, x_2, \cdots x_n$, such that:
$x_1 + x_2 + \cdots + x_n \geqs n$. 
\item Prove: $\sum_{i: x_i > 1/2} x_i \geqs n/2$. 
\end{itemize}
\end{frame}

\begin{frame}
\frametitle{A simple inequality}
\begin{itemize}
\item 
Break up the sum $x_1 + x_2 + \cdots + x_n$ into two parts, collecting all terms $i$ such that $x_i > 1/2$ and 
terms $i$ such that $x_i \leqs 1/2$.
\begin{center}
$\sum_i x_i = \sum_{i: x_i \leqs 1/2} x_i + \sum_{i: x_i > 1/2} x_i$
\end{center}
\item 
Let $k$ denote the number of terms such that $x_i \leqs 1/2$, so that:
\begin{center}
$ n \leqs \sum_i x_i \leqs k/2 + \sum_{i: x_i > 1/2} x_i$
\end{center}
\pause
\item So that: $\sum_{i: x_i > 1/2} x_i \geqs (n - k/2) \geqs n/2$.
\end{itemize}
\end{frame}

\begin{frame}
\frametitle{A simple inequality: Summary}
\begin{itemize}
\item Given the $n$ numbers, the {\color{red} mean} is $\geqs 1$. 
\item Maybe very few numbers have $x_i \approx 1$ $\cdots$ but \pause
\item Most of the {\color{blue} mass} is {\em around} $1$:
\begin{center}
$\sum_{i: x_i > 1/2} x_i \geqs n/2$
\end{center}
\pause
\item A concentration of measure result. 
\end{itemize}
\end{frame}

\begin{frame}
\frametitle{Inequality $2$}
\begin{itemize}
\item Setting: 
\begin{itemize}
\item
We are given a {\em digraph} $D = (V, A)$. 
\item 
Given a vertex $v$, let $i(v)$ denote the 
indegree of the vertex, and let $d(v)$ denote the total degree (in + out). 
\item 
Let $V_1 = \{v : i(v) \geqs d(v)/3\}$
\end{itemize}
\item Required to prove that: 
\begin{center}
$\sum_{v\in V_1} d(v) \geqs |A|/2$. 
\end{center}
\end{itemize}
\end{frame}

\begin{frame}
\frametitle{Inequality $2$}
\begin{itemize}
\item Another application of concentration of measure.
\item Collect facts:
\begin{itemize}
\item {\color{red} $\sum_{v\in V} d(v) = 2|A|$}
\item {\color{blue} $\sum_{v \in V} i(v) = |A|$}
\end{itemize}
\item Rewrite the {\color{blue} blue} inequality above as {\color{blue} $\sum_{v \in V} d(v)\cdot\frac{i(v)}{d(v)} = |A|$}. 
\item From here, we want to get: $\sum_{v\in V_1} d(v) \geqs |A|/2$.
\end{itemize}
\end{frame}

\begin{frame}
\frametitle{Inequality $2$: Use {\color{blue} blue} fact}
\begin{block}{Facts}
\begin{itemize}
\item {\color{red} $\sum_{v\in V} d(v) = 2|A|$}
\item {\color{blue} $\sum_{v \in V} i(v) = |A|$}
\end{itemize}
\end{block}
\begin{itemize}
%\item Facts:
%\begin{itemize}
%\item {\color{red} $\sum_{v\in V} d(v) = 2|A|$}
%\item {\color{blue} $\sum_{v \in V} i(v) = |A|$}
%\end{itemize}
\item Have: {\color{blue} $\sum_{v \in V} d(v)\cdot\frac{i(v)}{d(v)} 
= |A|$}
\item Want to have: $\sum_{v\in V_1} d(v) \geqs |A|/2$.
\item Separate out the vertices as $v\in V_1$ and $v\in V\backslash V_1$. 
\item For $v \in V \backslash V_1$, $i(v)/d(v) < 1/3$. For $v \in V_1$, $i(v)/d(v) \leqs 1$. 
\item {\color{blue} $|A| = \sum_{v \in V} d(v)\cdot\frac{i(v)}{d(v)} \leqs \sum_{v \in V_1} d(v) + \sum_{v\in V\backslash V_1} d(v)/3$}
\end{itemize}
\end{frame}





\end{document} 



\begin{frame}
\frametitle{}
\begin{itemize}
\item
\end{itemize}
\end{frame}
