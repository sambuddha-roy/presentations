\documentclass{beamer}
\usetheme{umbc4}    
\usepackage[english]{babel}
\usepackage[latin1]{inputenc}
\usepackage{times}
\usepackage[T1]{fontenc}
\usepackage{amssymb}
\usepackage{color}

\newcommand{\si}{\mathrm{sim}}
\newcommand{\dist}{\mathrm{dist}}
\newcommand{\lsh}{\mathrm{LSH}}



\newcommand{\mba}{\mathrm{MBA}}
\newcommand{\lst}{\mathrm{LST}}
\newcommand{\bfs}{\mathrm{BFS}}
\newcommand{\amgm}{{\mathrm{AM-GM}}}
\newcommand{\lp}{\mathrm{LP}}
\newcommand{\lr}{\mathrm{LR}}
\newcommand{\sig}{\mathrm{sign}}

\newcommand{\leqs}{\leqslant}
\newcommand{\geqs}{\geqslant}
\newcommand{\tcg}[1]{\textcolor{green}{#1}}
\newcommand{\tcb}[1]{\textcolor{blue}{#1}}
\newcommand{\tcr}[1]{\textcolor{red}{#1}}


\newcommand{\opt}{\mathrm{OPT}}
\newcommand{\eps}{\epsilon}
\newcommand{\tum}{\mathrm{TUM}}
\newcommand{\np}{\mathrm{NP}}
\newcommand{\dtime}{\mathrm{DTIME}}
\newcommand{\mis}{\mathrm{MIS}}
\newcommand{\maxcut}{{\tt MAXCUT}}

\newcommand{\calS}{{\cal S}}
\newcommand{\calA}{{\cal A}}
\newcommand{\calB}{{\cal B}}
\newcommand{\calC}{{\cal C}}
\newcommand{\calD}{{\cal D}}
\newcommand{\calE}{{\cal E}}
\newcommand{\calF}{{\cal F}}


\newcommand{\calT}{{\cal T}}
\newcommand{\calR}{{\cal R}}


\newcommand{\mstaff}{{\tt MaxStaff}}
\newcommand{\eat}[1] {}


\newcommand{\alrt}[1]{{\color{alert} #1}}
\newcommand{\alrtl}[1]{{\color{alert}\large #1}}
\newcommand{\alrtL}[1]{{\color{alert}\Large #1}}
\newcommand{\struc}[1]{{\color{structure} #1}}
\newcommand{\strucL}[1]{{\color{structure}\Large #1}}
\newcommand{\strucl}[1]{{\color{structure}\large #1}}


\newcommand{\paintblue}[1]{$\tikz[baseline]{
            \node[fill=blue!20,anchor=base] (t1)
            {$ #1$ };
        }$ }

\newcommand{\paintred}[1]{$\tikz[baseline]{
            \node[fill=red!20,anchor=base] (t1)
            {$ #1$ };
        } $}

\newcommand {\lovasz}{{Lov\'{a}sz}}

\newcommand{\textred}[1]{{\textcolor{red}{#1}}}
\newcommand{\textblue}[1]{{\textcolor{blue}{#1}}}


\usepackage{tikz,tkz-berge}
\setbeamertemplate{background canvas}[vertical shading][bottom=green!20,top=yellow!30]
\setbeamertemplate{itemize item}{\scriptsize\raise1.25pt\hbox{\donotcoloroutermaths$\blacktriangleright$}}


\title[IISc., July 7, 2016] % (optional, use only with long paper titles)
{Common Inequalities in Computer Science}

\author{Sambuddha Roy, LinkedIn}

\begin{document}

\begin{frame}
  \titlepage
\end{frame}

\begin{frame}
\frametitle{Contents}
\begin{itemize}
\item Why inequalities?
\item Used in: almost everywhere..
\pause
\begin{itemize}
\item Approximation Algorithms \pause 
\item Optimization \pause
\item Machine Learning \pause
\item Mathematical Analysis \pause
\item ...
\pause
\item (Can almost be called the backbone of mathematics..)
\end{itemize}
\end{itemize}
\end{frame}

\begin{frame}
\frametitle{See it in real life}
\begin{itemize}
\item So you have an algorithm - how do you prove it to be optimal (or close to optimal)?
\pause
\item You have to show that there is a {\em lower bound} for the resources that the algorithm 
takes.
\pause
\item Essentially - prove inequalities!
\end{itemize}
\end{frame}

\begin{frame}
\frametitle{A simple inequality}
\begin{itemize}
\item Setting: We are given $n$ positive real numbers $x_1, x_2, \cdots x_n$, such that:
$x_1 + x_2 + \cdots + x_n \geqs n$. 
\item Prove: $\sum_{i: x_i > 1/2} x_i \geqs n/2$. 
\end{itemize}
\end{frame}

\begin{frame}
\frametitle{A simple inequality}
\begin{itemize}
\item 
Break up the sum $x_1 + x_2 + \cdots + x_n$ into two parts, collecting all terms $i$ such that $x_i > 1/2$ and 
terms $i$ such that $x_i \leqs 1/2$.
\begin{center}
$\sum_i x_i = \sum_{i: x_i \leqs 1/2} x_i + \sum_{i: x_i > 1/2} x_i$
\end{center}
\item 
Let $k$ denote the number of terms such that $x_i \leqs 1/2$, so that:
\begin{center}
$ n \leqs \sum_i x_i \leqs k/2 + \sum_{i: x_i > 1/2} x_i$
\end{center}
\pause
\item So that: $\sum_{i: x_i > 1/2} x_i \geqs (n - k/2) \geqs n/2$.
\end{itemize}
\end{frame}

\begin{frame}
\frametitle{A simple inequality: Summary}
\begin{itemize}
\item Given the $n$ numbers, the {\color{red} mean} is $\geqs 1$. 
\item Maybe very few numbers have $x_i \approx 1$ $\cdots$ but \pause
\item Most of the {\color{blue} mass} is {\em around} $1$:
\begin{center}
$\sum_{i: x_i > 1/2} x_i \geqs n/2$
\end{center}
\pause
\item A concentration of measure result. 
\end{itemize}
\end{frame}

\begin{frame}
\frametitle{Inequality $2$}
\begin{itemize}
\item Setting: 
\begin{itemize}
\item
We are given a {\em digraph} $D = (V, A)$. 
\item 
Given a vertex $v$, let $i(v)$ denote the 
indegree of the vertex, and let $d(v)$ denote the total degree (in + out). 
\item 
Let $V_1 = \{v : i(v) \geqs d(v)/3\}$
\end{itemize}
\item Required to prove that: 
\begin{center}
$\sum_{v\in V_1} d(v) \geqs |A|/3$. 
\end{center}
\end{itemize}
\end{frame}

\begin{frame}
\frametitle{Inequality $2$}
\begin{itemize}
\item Another application of concentration of measure.
\item Collect facts:
\begin{itemize}
\item {\color{red} $\sum_{v\in V} d(v) = 2|A|$}
\item {\color{blue} $\sum_{v \in V} i(v) = |A|$}
\end{itemize}
\item Rewrite the {\color{blue} blue} inequality above as {\color{blue} $\sum_{v \in V} d(v)\cdot\frac{i(v)}{d(v)} = |A|$}. 
\item From here, we want to get: $\sum_{v\in V_1} d(v) \geqs |A|/3$.
\end{itemize}
\end{frame}

\begin{frame}
\frametitle{Inequality $2$: Use {\color{blue} blue} fact}
\begin{block}{Facts}
\begin{itemize}
\item {\color{red} $\sum_{v\in V} d(v) = 2|A|$}
\item {\color{blue} $\sum_{v \in V} i(v) = |A|$}
\end{itemize}
\end{block}
\begin{itemize}
\item Have: {\color{blue} $\sum_{v \in V} d(v)\cdot\frac{i(v)}{d(v)} 
= |A|$}
\item Want to have: $\sum_{v\in V_1} d(v) \geqs |A|/3$.
\item Separate out the vertices as $v\in V_1$ and $v\in V\backslash V_1$. 
\item For $v \in V \backslash V_1$, $i(v)/d(v) < 1/3$. For $v \in V_1$, $i(v)/d(v) \leqs 1$. 
\item {\color{blue} $|A| = \sum_{v \in V} d(v)\cdot\frac{i(v)}{d(v)} \leqs \sum_{v \in V_1} d(v) + \sum_{v\in V\backslash V_1} d(v)/3$}
\end{itemize}
\end{frame}

\begin{frame}
\frametitle{Inequality $2$: Use {\color{red} red} fact}
\begin{block}{Facts}
\begin{itemize}
\item {\color{red} $\sum_{v\in V} d(v) = 2|A|$}
\item {\color{blue} $\sum_{v \in V} i(v) = |A|$}
\end{itemize}
\end{block}
\begin{itemize}
\item {\color{blue} $|A| = \sum_{v \in V} d(v)\cdot\frac{i(v)}{d(v)} \leqs \sum_{v \in V_1} d(v) +$}{\color{red}$ \sum_{v\in V\backslash V_1} d(v)/3$}
\item i.e. $|A| \leqs \sum_{v \in V_1} d(v) + {\color{red} \frac{2|A|}{3}}$, which gives
\item $\sum_{v \in V_1} d(v) \geqs \frac{|A|}{3}$.
\end{itemize}
\end{frame}

\begin{frame}
\frametitle{Inequality $2$: Quiz Problem $1$ (Extra Credit) }
\begin{itemize}
\item We showed: 
\begin{center}
$\sum_{v \in V_1} d(v) \geqs |A|/3$.
\end{center}
\pause
\item Instead show that: 
\begin{center}
$\sum_{v \in V_1} d(v) \geqs |A|/{\color{red} 2}$. 
\end{center}
\end{itemize}
\end{frame}


\begin{frame}
\frametitle{Inequality $2$: Quiz Problem $1$ (Bonus Credit)}
\begin{itemize}
\item Show that: 
\begin{center}
$\sum_{v \in V_1} d(v) \geqs |A|$
\end{center}
\end{itemize}
\end{frame}


\begin{frame}
\frametitle{Comments}
\begin{itemize}
\item Why is this a ``concentration of measure'' result?
\item Note that $\sum_{v \in V} d(v) = 2\sum_{v \in V} i(v)$, so {\em on average}
$i(v)$ is roughly {\color{red} half} of $d(v)$. 
\item As before, we move slightly away from the mean and ask: how many vertices $v$
will satisfy $i(v) \geqs d(v)/3$? 
\item While we don't get a {\em count}, we get a result for the {\em mass}. 
\end{itemize}
\end{frame}

\begin{frame}
\frametitle{Concentration of measure}
\begin{itemize}
\item Chebyshev Inequalities
\item Markov Inequalities
\item Chernoff-Hoeffding bounds.
\end{itemize}
\end{frame}

\begin{frame}
\frametitle{Quiz Problem $2$}
\begin{itemize}
\item Recall Chebyshev's inequality: 
\begin{itemize}
\item Setting: let $X$ be a random variable with $E(X)= \mu$ and variance $\sigma^2 = var(X)$. 
\item Then: $P( | X - \mu | \geqs t ) \leqs \frac{\sigma^2}{t^2}$
\pause
\item Question: A one-sided version of Chebyshev
\pause
\item Prove that: 
\begin{center}
{\color{red}
$P( X - \mu \geqs t ) \leqs \frac{\sigma^2}{\sigma^2 + t^2}$. 
}
\end{center}
\end{itemize}
\end{itemize}
\end{frame}

\begin{frame}
\frametitle{Shifting Gears: Lagrangian Relaxations}
\begin{itemize}
\item Suppose we are given an optimization problem. 
\item What is an optimization problem? 
\begin{itemize}
\item Has an {\color{red} objective}
\item Has {\color{blue} constraints}
\end{itemize}
\pause
\item Some constraints are easy to satisfy \pause (called ``soft'' constraints)
\pause 
\item while some others are harder - hard constraints. 
\end{itemize}
\end{frame}

\begin{frame}
\frametitle{Lagrangian Relaxations}
\begin{itemize}
\item The big idea behind Lagrangian Relaxations is to ``relax'' the hard constraints by
{\em pulling them} into the objective. 
\end{itemize}
\end{frame}

\begin{frame}
\frametitle{An example}
An optimization problem:
\[ 
\min \sum_{i\in V} c_i x_i 
\]
\begin{eqnarray*}
\forall{(i, j) \in E} & {\color{blue} x_i + x_j \geqs 1}&\\
\forall{i \in V} & x_i \geqs 0&
\end{eqnarray*}

Looks familiar?
\end{frame}

\begin{frame}
\frametitle{Example modified}
Modified optimization problem:
\[ 
\min \sum_{i\in V} c_i x_i 
\]
\begin{eqnarray*}
\forall{(i, j) \in E} & {\color{blue} x_i + x_j \geqs 1}&\\
& {\color{red} \sum_{i \in V} x_i \geqs  k}&\\
\forall{i \in V} & x_i \geqs 0&
\end{eqnarray*}
Here, the {\color{blue} $x_i + x_j \geqs 1$} are the ``soft'' constraints, and 
{\color{red} $\sum_{i \in V} x_i \geqs  k$} is the ``hard'' constraint. 
\end{frame}

\begin{frame}
\frametitle{General Instance}
Optimization problem $O_1$:
\[ 
\min f(x)
\]
\begin{eqnarray*}
\mathrm{s.t.} &&\\
&{\color{blue} s(x) \leqs 0}&\\
&{\color{red} h(x) \leqs 0}&
\end{eqnarray*}
where, $s(x)$ stands for the soft constraints, and $h(x)$ for the hard constraints.
\end{frame}

\begin{frame}
\frametitle{Lagrangian Relaxation}
Relaxation $O_2 = \lr(\lambda)$ of $O_1$:
\[ 
\min f(x) + \lambda\cdot{\color{red} h(x)}
\]
\begin{eqnarray*}
\mathrm{s.t.} &&\\
&{\color{blue} s(x) \leqs 0}&\\
&\lambda \geqs 0&
\end{eqnarray*}
Here, $\lambda  \in [0, \infty)$ is called a {\color{blue} Lagrange Multiplier}. 
\begin{center}
Major Impact: ${\color{red} \mathbf{\lr(\lambda) \leqs O_1}}$!!!
%$\mathrm{Major Impact:} {\color{red} \mathbf{$\lr(\lambda) \leqs O_1}}$
\end{center}
\end{frame}

\begin{frame} 
\frametitle{H\"{o}lder's Inequality}
Setting: 
\begin{itemize}
\item
Vectors $\mathbf{x}= (x_1, \cdots, x_n), \mathbf{y} = (y_1, \cdots, y_n)$, where
$\mathbf{x}, \mathbf{y} \in \mathbb{R}^{n}_{\geqs 0}$.
\item Reals $p, q \in \mathbb{R}$ such that $\frac{1}{p} + \frac{1}{q} = 1$. 
\end{itemize}
Prove that:
\begin{center}
$(x_1^p + x_2^p + \cdots + x_n^p)^{1/p} \cdot (y_1^q + y_2^q + \cdots y_n^q)^{1/q} \geqs \mathbf{x}\cdot\mathbf{y}$
\end{center}
\end{frame}

\begin{frame}
\frametitle{H\"{o}lder's Inequality}
Have we seen (any special case of) this before?
With $p = q = 2$: 
\begin{center}
$(x_1^2 + x_2^2 + \cdots + x_n^2)^{1/2} \cdot (y_1^2 + y_2^2 + \cdots y_n^2)^{1/2} \geqs \mathbf{x}\cdot\mathbf{y}$
\end{center}
which is the Cauchy Schwarz inequality!
\end{frame}

\begin{frame}
\frametitle{Another H\"{o}lder's Inequality}
The following is also often called H\"{o}lder's Inequality (same conditions on $p, q$, non-negative
real numbers $x, y$):
\begin{center}
$\frac{x^p}{p} + \frac{y^q}{q} \geqs x y$
\end{center}
Proof is easy by $\mathrm{AM-GM}$. (Quiz Problem $3$). 
Call this the ``little'' H\"{o}lder's Inequality.
Condition for equality: $x^p = y^q$.

Now, from here to the first H\"{o}lder's Inequality? We will only discuss $n = 2$. 
\end{frame}

\begin{frame}
\frametitle{$n = 2$}
We want: 
\begin{center}
$(x_1^p + x_2^p)^{1/p}\cdot (y_1^q + y_2^q)^{1/q} \geqs \mathbf{x}\cdot \mathbf{y}$
\end{center}
First attempt: 
We can write down $2$ ``little'' H\"{o}lder Inequalities (one for $x_1, y_1$ another for $x_2, y_2$)
but that gives us:
\begin{center}
$\frac{x_1^p + x_2^p}{p}+\frac{y_1^q + y_2^q}{q} \geqs \mathbf{x}\cdot \mathbf{y}$
\end{center}
This is {\color{red} weaker} than what we want; why?
\end{frame}

\begin{frame}
\frametitle{Weaker?}
So far, we have:
\begin{center}
$\frac{x_1^p + x_2^p}{p}+\frac{y_1^q + y_2^q}{q} \geqs \mathbf{x}\cdot \mathbf{y}$
\end{center}
Let $z_1^p = x_1^p + x_2^p$ and $z_2^q = y_1^q + y_2^q$, then one application of the 
``little'' H\"{o}lder Inequality gives:
\begin{center}
$\frac{x_1^p + x_2^p}{p}+\frac{y_1^q + y_2^q}{q} \geqs z_1 \cdot z_2 = (x_1^p + x_2^p)^{1/p}\cdot (y_1^q + y_2^q)^{1/q}$
\end{center}
\end{frame}

\begin{frame}
\frametitle{Weaker to Stronger}
Terry Tao's suggestion: in order to milk most out of an inequality, apply it to a scenario where equality can
even conceivably hold. 

Also called the ``principle of maximum effectiveness''.

We realize that for the application of the ``little'' H\"{o}lder Inequality above, 
the equality condition of $x_1^p + x_2^p = y_1^q + y_2^q$ may be widely flouted. 
\end{frame}

\begin{frame}
\frametitle{Enter Lagrangian Multipliers}

Taking the cue from there, let's apply a Lagrangian Multiplier $\lambda$:
\begin{center} 
$\frac{x_1^p + x_2^p}{p}+{\color{blue}\lambda^q}\frac{y_1^q + y_2^q}{q} \geqs {\color{blue}\lambda}\mathbf{x}\cdot \mathbf{y}$
\end{center}

Now we can at least hope to achieve the case of equality by suitably choosing $\lambda = \lambda_0$.
\pause
\begin{center} 
$\lambda_0^q = \frac{x_1^p + x_2^p}{y_1^q + y_2^q}$
\end{center}
\end{frame}

\begin{frame}
\frametitle{Enter Lagrangian Multipliers}
\begin{center} 
${\color{blue}\frac{x_1^p + x_2^p}{p}+{\color{blue}\lambda^q}\frac{y_1^q + y_2^q}{q}} \geqs {\color{blue}\lambda}\mathbf{x}\cdot \mathbf{y}$
\end{center}
\begin{center}
$\lambda_0^q = \frac{x_1^p + x_2^p}{y_1^q + y_2^q}$
\end{center}
Applying $1/p + 1/q = 1$, 
the LHS ${\color{blue}\frac{x_1^p + x_2^p}{p}+{\color{blue}\lambda_0^q}\frac{y_1^q + y_2^q}{q}}$ simplifies to 
$(x_1^p + x_2^p)$ and we have: $(x_1^p + x_2^p) \geqs \lambda_0 \mathbf{x}\cdot \mathbf{y}$.
\end{frame}

\begin{frame}
\frametitle{Enter Lagrangian Multipliers}
The LHS ${\color{blue}\frac{x_1^p + x_2^p}{p}+{\color{blue}\lambda_0^q}\frac{y_1^q + y_2^q}{q}}$ simplifies to$(x_1^p + x_2^p)$ and we have: $(x_1^p + x_2^p) \geqs \lambda_0 \mathbf{x}\cdot \mathbf{y}$.

One last application of $1/p + 1/q = 1$ gives 
\begin{center}
$(x_1^p + x_2^p)^{1/p}\cdot (y_1^q + y_2^q)^{1/q} \geqs \mathbf{x}\cdot \mathbf{y}$
\end{center}
and we are done!
\pause

The inequality for $n$ quantities is similar.
\end{frame}

\begin{frame}
\frametitle{Ode to beautiful inequalities}
The world of inequalities is mesmerizing:
\begin{itemize}
\item 
The mother of all inequalities: Cauchy Schwarz Inequality
$(x_1^2 + x_2^2 + \cdots + x_n^2)^{1/2}(y_1^2 + y_2^2 + \cdots y_n^2)^{1/2} \geqs \mathbf{x}\cdot\mathbf{y}$.
\end{itemize}
\begin{center}
\includegraphics[height=1.5in]{cauchySchwarzMasterClass.jpeg}
\end{center}
\end{frame}

\begin{frame}
\frametitle{Ode to beautiful inequalities}
Sample applications:
\begin{enumerate}
\item $(x^2 + y^2 + z^2) \geqs (xy + yz + zx)$; restatement (AM-GM for $3$ numbers): $x^3 + y^3 + z^3 \geqs 3xyz$.
\item Loomis-Whitney Inequality: Compares the volume of a set in terms of the volumes of the projections of that 
set onto lower dimensional subspaces. 

For $3$ dimensions: given a set $A \in \mathbb{R}^3$ and its projections onto the axes $A_x, A_y, A_z$ it holds that:
\begin{center}
$|A| \leqs |A_x|^{1/2}|A_y|^{1/2}|A_z|^{1/2}$
\end{center}
\end{enumerate}
\end{frame}

\begin{frame}
\frametitle{Ode to beautiful inequalities}
\begin{enumerate}
\item Harker-Kasper Inequality in Crystallography
\pause
\item Cram\'{e}r-Rao Lower Bound in Statistics
\pause
\item Jensen's Inequality in Convex Analysis, Expectation Maximization methods.
\pause
\item Many, many, many more...
\end{enumerate}
\end{frame}

\begin{frame}
\frametitle{And more...}
\bf{Carleman's Inequality}:
\begin{center}
$\sum_{k=1}^{\infty} (a_1 a_2 \cdots a_k)^{1/k} \leqs e \sum_{k=1}^{\infty} a_k$.
\end{center}
\end{frame}

\begin{frame}
\begin{center}
{\bf THANK YOU}
\end{center}
\end{frame}


\end{document} 


\begin{frame}
\frametitle{}
\begin{itemize}
\item
\end{itemize}
\end{frame}
